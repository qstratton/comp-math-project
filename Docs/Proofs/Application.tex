\documentclass{report}
\usepackage[utf8]{inputenc}
\usepackage{amsthm}
\usepackage{amsmath, mathtools, amsfonts,amssymb}
\usepackage{graphicx}
\usepackage{verbatim}
\usepackage{fancyhdr}
\usepackage{subfigure}
\usepackage{float}
% Code
\usepackage{listings} 
\usepackage{algorithm}
\usepackage{algpseudocode}
% Margins
\usepackage{geometry}
%\geometry{margin=1in} %% 

% Graphics
\usepackage{tikz}
\usetikzlibrary{matrix} % matrices
\usepackage{tikz-qtree} % Simple trees

% Theorems/etc.
\newtheorem{pic}{Figure}
\numberwithin{pic}{section}
\newtheorem{lem}{Lemma}
\numberwithin{lem}{section}
\newtheorem{thm}{Theorem}
\numberwithin{thm}{section}
\newtheorem{cor}{Corollary}
\numberwithin{cor}{section}

\theoremstyle{definition}
\newtheorem{ex}{Example}
\numberwithin{ex}{section}
\newtheorem{defn}{Definition}
\numberwithin{defn}{section}
\theoremstyle{definition}
\newtheorem{prob}{Problem}

\theoremstyle{remark}
\newtheorem*{con}{Conjecture}
\newtheorem{rem}{Remark}
\newtheorem*{cex}{Counterexample}
\newtheorem*{ts}{T.S.}

%%% COMMANDS %%%
% Sets
\newcommand{\set}[1]{\ensuremath{\left\{ #1\right\}}} % write sets
\newcommand{\e}{\ensuremath{\epsilon}} % Epsilon
\newcommand{\R}{\ensuremath{\mathbb{R}}} % Real Numbers
\newcommand{\N}{\ensuremath{\mathbb{N}}} % Natural numbers
\newcommand{\Q}{\ensuremath{\mathbb{Q}}} % Rationals
\newcommand{\I}{\ensuremath{\mathbb{I}}} % Irrational Numbers
\newcommand{\Z}{\ensuremath{\mathbb{Z}}} % Integers
% Easier Delimiters?
\newcommand{\lr}[2]{\ensuremath{\left#1 #2 \right #1}}
% Absolute Value
\newcommand{\abs}[1]{\ensuremath{\left| #1 \right|}}
% Landau Notation
\newcommand{\Oh}{\ensuremath{\mathcal{O}}} %%% IN MATH MODE
\newcommand{\oh}{\ensuremath{\mathcal{o}}} %%% IN MATH MODE
% Display style fractions
\newcommand{\Frac}[2]{\displaystyle \frac{#1}{#2}}
% Display style limits
\newcommand{\Lim}[2]{\displaystyle \lim_{#1}{#2}}

%%% Preference Stuff %%%
\setcounter{section}{-1}
\renewcommand\qedsymbol{{$\blacksquare$}}
% Enumerate
\renewcommand{\labelenumi}{(\alph{enumi})}
\renewcommand{\labelenumii}{\roman{enumii}}
% change proof environment
\renewcommand*{\proofname}{Pf}
% Line Spacing
\renewcommand{\baselinestretch}{1.5}
% Indentation
\newlength\tindent
\setlength{\tindent}{\parindent}
\setlength{\parindent}{0pt}
\renewcommand{\indent}{\hspace*{\tindent}}

\begin{document}
\section{Origin of the problem and its applications}
In this part of the project we explain the motivation behind our ideas and consider the applications of the results.\\
First, we consider the standard elliptic equation in two dimensions
\begin{align*}
-\Delta u &= f ~\text{ in } \Omega\\
u&=0 ~\text{ on } \partial\Omega.
\end{align*}
Then the discretised equation on a grid is
\begin{align*}
-\Delta_h u_{ij} &= f_{ij} ~\forall\,(x_i,y_i)\in \Omega_h,~f_{ij}=f(x_i,y_j)\\
u_{ij}&=0 ~\forall\,(x_i,y_i)\in \partial\Omega_h,
\intertext{where we use the second order central differencing to represent the Laplace operator}
-\Delta_h u_{ij}= \Frac 1{h^2}\begin{pmatrix}
0 & -1 & 0 \\ 
-1 & 4 & -1 \\ 
0 & -1 & 0
\end{pmatrix}u_{ij}&= \Frac 1{h^2} \big(-u_{ij+1}-u_{i-1j}+4 u_{ij}-u_{i+1j}-u_{ij_1}\big)
\end{align*}
If we assume that the domain $\Omega$ is a square and ordering the grid points from left to right and bottom to top yields the following matrix of the system $A\in \R^{(h-1)^{-2}\times (h-1)^{-2}}$:
\[A=h^2\begin{pmatrix}
4 & -1 & 0 & \dots & 0 & -1 &  &  \\ 
-1 & 4 & -1 &  &  &  & \ddots &  \\ 
0 & -1 & 4 & \ddots &  &  &  & -1 \\ 
\vdots  &  & \ddots & \ddots &  &  &  & 0\\ 
0 &  &  &  &  &  &  & \vdots \\ 
-1 &  &  &  &  & \ddots & \ddots & 0 \\ 
 & \ddots &  &  &  & \ddots & 4 & -1 \\ 
 &  & -1 & 0 & \dots & 0 & -1 & 4
\end{pmatrix} \]
And the problem we need to solve is the linear system $Au=f$.\\
Therefore it is important to understand how the structure of $A$ affects the structure of the $QR$ decomposition.
Note that in this case the highest and lowest off-diagonal has a distance of order   $h$ from the diagonal.\\

Another example where these banded matrices show up is in the following:
Consider the parabolic equation in two dimensions
\[\frac{\partial u}{\partial t}=\sigma \Delta u,~ 0\leq x\leq X,\, 0\leq y\leq Y\, 0\leq t\leq T\]
with Dirichlet boundary condition and a given initial data $u(x,y,0)=U^0(x,y)$.\\
For the numerical implementation we consider the implicit Crank-Nicolson scheme
\begin{align*}
&-\frac{\mu_x}{2}\big(U_{j-1,l}^{n+1}+ U_{j+1,l}^{n+1}\big) -\frac{\mu_y}{2}\big(U_{j,l-1}^{n+1}+U_{j,l+1}^{n+1}\big)+\big(1+\mu_x+\mu_y\big)U_{j,l}^{n+1}\\
=&\frac{\mu_x}{2}\big(U_{j-1,l}^{n}+ U_{j+1,l}^{n}\big) +\frac{\mu_y}{2}\big(U_{j,l-1}^{n}+U_{j,l+1}^{n}\big)+\big(1-\mu_x-\mu_y\big)U_{j,l}^{n},
\end{align*}
for $0\leq j\leq J_x$, $0\leq l\leq J_y$ and $n>0$.
Again we can rewrite this as a linear system $AU^{n+1}=U^n$, where 
\[A=\begin{pmatrix}
1+\mu_x+\mu_y & -\frac{\mu_x}{2} & 0 & \dots & 0 & -\frac{\mu_y}{2} &  &  \\ 
-\frac{\mu_x}{2} & 1+\mu_x+\mu_y & -\frac{\mu_x}{2} &  &  &  & \ddots &  \\ 
0 & -\frac{\mu_x}{2} & 1+\mu_x+\mu_y & \ddots &  &  &  & -\frac{\mu_y}{2} \\ 
\vdots  &  & \ddots & \ddots &  &  &  & 0\\ 
0 &  &  &  &  &  &  & \vdots \\ 
-\frac{\mu_y}{2} &  &  &  &  & \ddots & \ddots & 0 \\ 
 & \ddots &  &  &  & \ddots & 1+\mu_x+\mu_y & -\frac{\mu_x}{2} \\ 
 &  & -\frac{\mu_y}{2} & 0 & \dots & 0 & -\frac{\mu_x}{2} & 1+\mu_x+\mu_y
\end{pmatrix} \]
and $A\in\R^{(J_x-1)(J_y-1)\times (J_x-1)(J_y-1)}$ where the highest and lowest off-diagonal band have the distance $J_x-1$ from the diagonal.
\begin{rem}
In the case of three dimension, we would obtain one more non-zero sub/super-diagonal, now with a distance of $\mathcal{O}(J_x*J_y)$ from the diagonal.
\end{rem}
\end{document}