\documentclass{report}
\usepackage[utf8]{inputenc}
\usepackage{amsthm}
\usepackage{amsmath, mathtools, amsfonts,amssymb}
\usepackage{graphicx}
\usepackage{verbatim}
\usepackage{fancyhdr}
% Code
\usepackage{listings} 
\usepackage{algorithm}
\usepackage{algpseudocode}
% Margins
\usepackage{geometry}
%\geometry{margin=1in} %% 
% Graphics
\usepackage{tikz}
\usetikzlibrary{matrix} % matrices
\usepackage{tikz-qtree} % Simple trees
% Theorems/etc.
\newtheorem{pic}{Figure}
\numberwithin{pic}{section}
\newtheorem{lem}{Lemma}
\numberwithin{lem}{section}
\newtheorem{thm}{Theorem}
\numberwithin{thm}{section}
\newtheorem{cor}{Corollary}
\numberwithin{cor}{section}
\theoremstyle{definition}
\newtheorem{ex}{Example}
\numberwithin{ex}{section}
\newtheorem{defn}{Definition}
\numberwithin{defn}{section}
\theoremstyle{definition}
\newtheorem{prob}{Problem}
\theoremstyle{remark}
\newtheorem*{con}{Conjecture}
\newtheorem{rem}{Remark}
\newtheorem*{cex}{Counterexample}
\newtheorem*{ts}{T.S.}
%%% COMMANDS %%%
% Sets
\newcommand{\set}[1]{\ensuremath{\left\{ #1\right\}}} % write sets
\newcommand{\e}{\ensuremath{\epsilon}} % Epsilon
\newcommand{\R}{\ensuremath{\mathbb{R}}} % Real Numbers
\newcommand{\C}{\ensuremath{\mathbb{C}}} % Real Numbers
\newcommand{\N}{\ensuremath{\mathbb{N}}} % Natural numbers
\newcommand{\Q}{\ensuremath{\mathbb{Q}}} % Rationals
\newcommand{\I}{\ensuremath{\mathbb{I}}} % Irrational Numbers
\newcommand{\Z}{\ensuremath{\mathbb{Z}}} % Integers
% Easier Delimiters?
\newcommand{\lr}[2]{\ensuremath{\left#1 #2 \right #1}}
% Absolute Value
\newcommand{\abs}[1]{\ensuremath{\left| #1 \right|}}
% Landau Notation
\newcommand{\Oh}{\ensuremath{\mathcal{O}}} %%% IN MATH MODE
\newcommand{\oh}{\ensuremath{\mathcal{o}}} %%% IN MATH MODE
% Display style fractions
\newcommand{\Frac}[2]{\displaystyle \frac{#1}{#2}}
% Display style limits
\newcommand{\Lim}[2]{\displaystyle \lim_{#1}{#2}}
%%% Preference Stuff %%%
\setcounter{section}{-1}
\renewcommand\qedsymbol{{$\blacksquare$}}
% Enumerate
\renewcommand{\labelenumi}{(\alph{enumi})}
\renewcommand{\labelenumii}{\roman{enumii}}
% change proof environment
\renewcommand*{\proofname}{Pf}
% Line Spacing
\renewcommand{\baselinestretch}{1.5}
% Indentation
\newlength\tindent
\setlength{\tindent}{\parindent}
\setlength{\parindent}{0pt}
\renewcommand{\indent}{\hspace*{\tindent}}
% Set title
\title{Computational Math Project\\
  {\large Illinois Institute of Technology}
}
\author{
  Miles Bakenhus 
  \and
  Ahmed Lodhika 
  \and
  Gunjan Sharma 
  \and
  Quinn Stratton 
  \and
  Jan-Eric Sulzbach 
}
\date{\today}
\begin{document}
\fancyhead[l]{}
\fancyhead[c]{}
\fancyhead[r]{}
% \pagestyle{fancy}
%\maketitle
%\tableofcontents
%\section{Introduction}
%\textbf{\emph{Pf.}}
\begin{proof}
  
Let $A = QR$ for $A,Q \in \C^{m \times m}$, $R \in \C^{m \times n}$, unitary $Q$, and upper triangular $R$. If $A$ has bandwidth $2p +1$ then for $i - j > p$, 
	$$ 0 = a_{ij} = \sum_{k = 1}^{m} q_{ik}r_{kj}$$
Since $R$ is upper triangular, $r_{kj} = 0$ for $k > j$. Then when $i > j + p$,
	$$ 0 = a_{ij} = \sum_{k = 1}^{j} q_{ik}r_{kj}$$
so $q_{ij} = 0$ . Hence 
	\begin{equation} \label{eq:1}
		q_{j} = \begin{pmatrix} 
			q_{1,j} \\
			q_{2,j} \\
			\vdots \\
			q_{j+p,j} \\
			0 \\
			\vdots \\
			0 \\
		\end{pmatrix}
	\end{equation}

From (\ref{eq:1}) when $i+p < j - p -1$ (i.e., $j - i > 2p + 1$):
	$$
		r_{ij} = q_i^* a_j =\begin{pmatrix}
			q_{1,i} & q_{2,i} &	\dots & q_{i+p,i} & 0 & \dots &	0 
		\end{pmatrix}\begin{pmatrix} 
		0 \\
		\vdots \\
		0 \\
		a_{j-p-1,j} \\
		\vdots \\
		a_{j+p+1,j} \\
		0 \\
		\vdots \\
		0 \\
		\end{pmatrix} = 0
	$$
Hence $R$ is upper triangular with its only nonzero entries in the diagonal and $2p$ super diagonals.
\end{proof}
\end{document}
